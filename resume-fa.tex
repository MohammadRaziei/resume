%-------------------------
% Resume LaTeX Template for Mohammad Raziei
% Author : Mohammad Raziei
% Based on various LaTeX resume templates
% License : MIT

% This resume template is designed to showcase professional experience and skills
% in a clean, organized format. Feel free to modify it to suit your needs.
%-------------------------

%------------PACKAGES----------------
\documentclass[a4paper,11pt]{article}

\usepackage{verbatim} % reimplements the "verbatim" and "verbatim*" environments

\usepackage{titlesec} % provides an interface to sectioning commands i.e. custom elements

\usepackage{color} % provides both foreground and background color management

\usepackage{enumitem} % provides control over enumerate, itemize and description

\usepackage{fancyhdr} % provides extensive facilities for constructing headers, footers and also controlling their use

\usepackage{multicol,array}
\usepackage{tabularx} % defines an environment tabularx, extension of "tabular" with an extra designator x, paragraph like column whose width automatically expands to fill the width of the environment

%\usepackage{latexsym} % provides mathematical symbols

%\usepackage{marvosym} % provides martin vogel's symbol font which contains various symbols

\usepackage[empty]{fullpage} % sets margins to one inch and removes headers, footers etc..

\usepackage[hidelinks]{hyperref} % removes color and shadow of hyperlinks

\usepackage[normalem]{ulem} % provides "\ul" (uline) command which will break at line breaks

\usepackage[english]{babel} % provides culturally determined typographical rules for wide range of languages

\usepackage{ragged2e}

%\usepackage{kpfonts}
\usepackage{amsmath,amsthm,mathtools}
\usepackage{fontawesome5}
\usepackage{enumitem}

%-----------------------------------------

%\input{glyphtounicode} % converts glyph names to unicode
%\pdfgentounicode=1 % ensures pdfs generated are ats readable

%-----------------------------------------
\newlist{subitemize}{itemize}{3}
\setlist[subitemize]{label=$\hookleftarrow$,before={\vspace{-4pt}\raggedright\justifying}, resume,leftmargin=*}

%----------FONT OPTIONS-------------------
%\usepackage[default]{sourcesanspro}
\usepackage{graphicx}
\usepackage{wasysym}
\usepackage{textcomp} % uses the font source sans pro
\urlstyle{same} % changes url font from default urlfont to font being used by the document

%-----------------------------------------
\usepackage{xepersian}
\settextfont[Scale=1.05]{XB Niloofar} % یا هر فونت فارسی دیگری که نصب کرده‌اید


%----------MARGIN OPTIONS-----------------
\pagestyle{fancy} % set page style to one configured by fancyhdr
\fancyhf{} % clear all header and footer fields

\renewcommand{\headrulewidth}{0in} % sets thickness of linerule under header to zero
\renewcommand{\footrulewidth}{0in} % sets thickness of linerule over footer to zero

\setlength{\tabcolsep}{0in} % sets thickness of column separator in tables to zero

% origin of the document is one inch from the top and from and the left
% oddsidemargin and evensidemargin both refer to the left margin
% right margin is indirectly set using oddsidemargin
\addtolength{\oddsidemargin}{-0.5in}
\addtolength{\topmargin}{-0.5in}

\addtolength{\textwidth}{1.0in} % sets width of text area in the page to one inch
\addtolength{\textheight}{1.0in} % sets height of text area in the page to one inch

\raggedbottom{} % makes all pages the height of current page, no extra vertical space added
\raggedright{} % makes all pages the width of current page, no extra horizontal space added
%------------------------------------------


%--------SECTIONING COMMANDS---------
% \titleformat{<command>}
%   [<shape>]{<format>}{<label>}{<sep>}
%     {<before-code>}[<after-code>]

% command is the sectioning command to be redefined
% shape is the style of the font; scshape stands for small caps style
% format is the format to be applied to whole title- label and text; absent here
% label defines the label
% sep is the horizontal separation between label and title body
% before-code is the code to be executed before
% after-code is the code to be executed after

\titleformat{\section}
{\scshape\large}{}
{0em}{\color{blue}}[\color{black}\titlerule\vspace{0pt}]
%-------------------------------------


%--------REDEFINITIONS----------------
% redefines the style of the bullet point
\renewcommand\labelitemii{$\vcenter{\hbox{\tiny$\bullet$}}$}

% redefines the underline depth to 2pt
\renewcommand{\ULdepth}{2pt}
%-------------------------------------


%--------CUSTOM COMMANDS--------------
%\vspace{} defines a vertical space of given size, modifying this in custom commands can help stretch or shrink resume to remove or add content

% resumeItem renders a bullet point
\newcommand{\resumeItem}[1]{
\item{\small{#1}}
}
\newcommand{\resumeItemScore}[2]{\item[]\small#1\hfill#2}


% commands to start and end itemization of resumeItem, rightmargin set to 0.11in to avoid the overflow of resumetItem beyond whatever resumeItemHeading is being used
\newcommand{\resumeItemListStart}{\begin{itemize}[rightmargin=0.11in]}
\newcommand{\resumeItemListEnd}{\end{itemize}}

% resumeSectionType renders a bolded type to be used under a section, used as skill type here, middle element is used to keep ":"s in the same vertical line
\newcommand{\resumeSectionType}[3]{
\item\begin{tabular*}{0.96\textwidth}[t]{
p{0.15\linewidth}p{0.02\linewidth}p{0.81\linewidth}
}
\textbf{#1} & #2 & #3
\end{tabular*}\vspace{-2pt}
}

% resumeTrioHeading renders three elements in three columns with second element being italicized and first element bolded, can be used for projects with three elements
\newcommand{\resumeTrioHeading}[3]{
\item\small{
\begin{tabular*}{0.96\textwidth}[t]{
l@{\extracolsep{\fill}}c@{\extracolsep{\fill}}r
}
\textbf{#1} & \textit{#2} & #3
\end{tabular*}
}
}

\newcommand{\resumePantaHeading}[5]{
\item\small{
\begin{tabular*}{0.96\textwidth}[t]{
l@{\extracolsep{\fill}}c@{\extracolsep{\fill}}r
}
\textbf{#1} & #2 & #3 & #4 & #5
\end{tabular*}
}
}

% resumeQuadHeading renders four elements in a two columns with the second row being italicized and first element of first row bolded, can be used for experience and projects with four elements
\newcommand{\resumeQuadHeading}[4]{
\item
\begin{tabular*}{0.96\textwidth}[t]{l@{\extracolsep{\fill}}r}
\textbf{#1} & #2 \\
\textit{\small#3} & \textit{\small #4} \\
\end{tabular*}
}

% resumeQuadHeadingChild renders the second row of resumeQuadHeading, can be used for experience if different roles in the same company need to added
\newcommand{\resumeQuadHeadingChild}[2]{
\item
\begin{tabular*}{0.96\textwidth}[t]{l@{\extracolsep{\fill}}r}
\textbf{\small#1} & {\small#2} \\
\end{tabular*}
}


% commands to start and end itemization of resumeQuadHeading, lefmargin for left indent of 0.15in for resumeItems
\newcommand{\resumeHeadingListStart}{
\begin{itemize}[leftmargin=0.15in, label={}]
}
\newcommand{\resumeHeadingListEnd}{\end{itemize}}
%-------------------------------------------
\newcommand{\vcenteredinclude}[2][none]{\begingroup
\setbox0=\hbox{\includegraphics[#1]{#2}}%
\parbox{\wd0}{\box0}\endgroup}

%__________________RESUME____________________
% You can rearrange sections in any order you may prefer
\begin{document}

%-----------CONTACT DETAILS------------------
% Make sure all the details are correct, you can add more links in the first row of second column if needed

\begin{tabular*}{\textwidth}{r@{\extracolsep{\fill}}l}

\textbf{\Huge محمد رضیئی فیجانی \vspace{2pt}} % Translated Name

&
\vcenteredinclude[width=2.5cm]{images/profile}
\\
\multicolumn{2}{r}{
\begin{tabular}{l@{\quad|\quad}l@{\quad|\quad}l}
\faLocationArrow\hspace{3pt} مکان: تهران، ایران  % row = 2, col = 1
&
\faMobile\hspace{3pt} موبایل: \lr{+989105093143}   % row = 2, col = 1
&
\faTelegram\hspace{3pt} تلگرام: \href{https://t.me/mohammadraziei}{\uline{\lr{@mohammadraziei}}}
\end{tabular}
}
\\
\faLinkedin\hspace{3pt} \uline{\url{https://www.linkedin.com/in/mohammadraziei/}} &
\faEnvelope\hspace{3pt} ایمیل: \hspace{8pt} \href{mailto:mohammadraziei1375@gmail.com}{\uline{\lr{mohammadraziei1375@gmail.com}}} % row = 2, col = 2
\\
\faGithub\hspace{3pt} \uline{\url{https://www.github.com/mohammadraziei/}} &
\faGlobe\hspace{3pt} وب‌سایت: \uline{\url{https://mohammadraziei.github.io}}
\\
\end{tabular*}
%--------------------------------------------




%-----------SUMMARY--------------------------
% Keep this short, simple and straigth to point
\justifying
\section{درباره‌ من} % Translated Section Title
\small{
دانشمند و مهندس داده بسیار با انگیزه با پایه قوی در پردازش سیگنال (تصویر، پردازش زبان طبیعی (\lr{NLP})، صوت)، موازی‌سازی و محاسبات با عملکرد بالا (\lr{HPC}). دارای مدرک کارشناسی در سیستم‌های مخابراتی و مهندسی کامپیوتر (دانشگاه صنعتی امیرکبیر) و کارشناسی ارشد در مهندسی پزشکی (دانشگاه صنعتی شریف) با تجربه عملی در بازسازی تصویر \lr{MRI}. اکنون دانشجوی دکترا در رشته مخابرات در دانشگاه صنعتی شریف (پذیرفته شده از طریق برنامه نخبگان) هستم. تحقیقات دکترای من به بررسی کاربرد سطوح هوشمند قابل پیکربندی (\lr{RIS}) برای بهینه‌سازی محیط‌های بی‌سیم و توسعه الگوریتم‌های پیشرفته پردازش سیگنال برای مکان‌یابی با دقت بالا در سیستم‌های ارتباطی نسل آینده می‌پردازد. مسلط به توسعه بک‌اند (\lr{backend development}) و پیاده‌سازی الگوریتم‌های پیچیده، در محیط‌های همکاری تیمی رشد می‌کنم، مهارت‌های ارتباطی قوی دارم و به یادگیری مستمر برای پیشبرد نوآوری و حل چالش‌های پیچیده متعهد هستم.
}
%--------------------------------------------
%-----------EDUCATION-------------------------
% Mention your CGPA, if its good, in the first row of second column

\section{تحصیلات} % Translated Section Title
\resumeHeadingListStart{}

\resumeQuadHeading{دانشگاه صنعتی شریف}{\faGraduationCap\hspace{3pt} تهران، ایران}
{۱۴۰۳ -- اکنون}{\textbf{دکتری} مهندسی سیستم‌های مخابراتی - دانشکده مهندسی برق}
\begin{subitemize}
	\item\textbf{موضوع رساله:}
		\textit{پردازش سیگنال سطوح هوشمند بازپیکربارپذیکر (\lr{RIS})}.
	\item[]\textbf{توضیحات:}
		پس از اتمام دوره خدمت سربازی، در سال ۱۴۰۳ بدون شرکت در کنکور و از طریق بنیاد ملی نخبگان وارد دوره دکتری شدم. تحقیقات من بر روی فناوری سطوح هوشمند تنظیم شونده (\lr{RIS}) متمرکز است.
	\item[]\textbf{استاد راهنما:} دکتر م. فخارزاده (\href{https://sharif.edu/~fakharzadeh/}{\uline{وب‌سایت}})
\end{subitemize}
\vspace{.5em}

\resumeQuadHeading{دانشگاه صنعتی شریف}{\faGraduationCap\hspace{3pt} تهران، ایران}
{۱۳۹۸ -- ۱۴۰۱}{\textbf{کارشناسی ارشد} مهندسی پزشکی - دانشکده مهندسی برق}
\begin{subitemize}
	\item\textbf{عنوان پایان‌نامه:}
		\textit{بازسازی تصاویر \lr{MRI} زیرنمونه‌برداری شده}.
	\item[]\textbf{توضیحات:}
		تسریع در تصویربرداری \lr{MRI} با استفاده از تکنیک‌های \textit{سنجش فشرده (\lr{compressed sensing})} و تصویربرداری موازی برای بازسازی زیرنمونه‌برداری شده. دستیابی به افزایش سرعت ۱۰ برابری با حفظ کیفیت بالای تصویر. دفاع موفق (اوایل سال ۱۴۰۱)؛ محرمانه طبقه‌بندی شده (تا ۵ سال) برای انتشار و ثبت اختراع.
	\item[]\textbf{استاد راهنما:} دکتر ب. وثوقی وحدت (\href{https://ee.sharif.edu/~vahdat/}{\uline{وب‌سایت}})
	\end{subitemize}
\vspace{.5em}

\resumeQuadHeading{دانشگاه صنعتی امیرکبیر}{\faGraduationCap\hspace{3pt} تهران، ایران}
{۱۳۹۶ -- ۱۳۹۹}{\textbf{کارشناسی} مهندسی نرم‌افزار - دانشکده مهندسی کامپیوتر}
\begin{subitemize}
	\item[]\textbf{توضیحات:}
		به دلیل شرایط ممتاز حاصل از معدل دوره کارشناسی و رتبه در کنکور سراسری (۴۶۱)، علاوه بر رشته مهندسی برق، از طریق بنیاد ملی نخبگان دانشگاه، در رشته کامپیوتر نیز به ادامه تحصیل پرداختم.
	\end{subitemize}
\vspace{.5em}

\resumeQuadHeading{دانشگاه صنعتی امیرکبیر}{\faGraduationCap\hspace{3pt} تهران، ایران}
{۱۳۹۴ -- ۱۳۹۸}{\textbf{کارشناسی} مهندسی سیستم‌های مخابراتی - دانشکده مهندسی برق}
\begin{subitemize}
	\item\textbf{عنوان پایان‌نامه:} \textit{الگوریتم مبتنی بر یادگیری تقویتی برای کنترل وسایل نقلیه خودران}.
    \item[]\textbf{جزئیات پروژه:} توسعه \textit{\lr{Gym-Prescan}}، یک بسته پایتون که \lr{OpenAI Gym} را با پلتفرم شبیه‌سازی \lr{PreScan} مرتبط می‌کند و امکان انجام آزمایش‌های یادگیری تقویتی برای کنترل وسایل نقلیه خودران را فراهم می‌آورد. این ابزار هسته اصلی کار پایان‌نامه را تشکیل داد.
	\item[]\textbf{جایزه:}
		این پایان‌نامه در پایان سال ۱۳۹۸ رتبه اول صنعت را کسب کرد. این رویداد با ارزیابی هیئتی از اساتید برق و داوران صنعت برگزار شد.
	\item[]\textbf{اساتید راهنما:} دکتر و. پوراحمدی (\href{https://aut.ac.ir/cv/2519/VAHID%20POURAHMADI}{\uline{وب‌سایت}})، دکتر ح. امین‌داور (\href{https://aut.ac.ir/cv/2200/Hamidreza-Amindavar}{\uline{وب‌سایت}})،
\end{subitemize}
\resumeHeadingListEnd{}
%---------------------------------------------
\section{برخی نمرات} % Translated Section Title
%\resumeHeadingListStart{}
%
\resumeHeadingListStart{}
\resumeQuadHeading{کارشناسی ارشد}{}{}{}\\\vspace{-1.5em} % Translated "Master"

\begin{multicols}{2}
\resumeItemListStart{}
\resumeItemScore{پردازش سیگنال گراف (\lr{GSP})}{۲۰}
\resumeItemScore{برنامه‌نویسی موازی}{۲۰}
\resumeItemScore{یادگیری عمیق}{۲۰}
\resumeItemScore{تصویربرداری پزشکی}{۱۸.۵}
\resumeItemScore{سنجش فشرده (\lr{CS})}{۱۸.۳}
\resumeItemScore{تحلیل و پردازش تصاویر پزشکی (\lr{MIAP})}{۱۸}
\resumeItemScore{جداسازی کور منابع (\lr{BSS})}{۱۷.۶}
\resumeItemScore{فیلتر تطبیقی}{۱۷}
\resumeItemListEnd{}
\end{multicols}


\resumeQuadHeading{کارشناسی}{}{}{}\\\vspace{-1.5em} % Translated "Bachelor"
\begin{multicols}{2}
\resumeItemListStart{}
\resumeItemScore{پردازش سیگنال دیجیتال (\lr{DSP})}{۲۰}
\resumeItemScore{برنامه‌نویسی کامپیوتر}{۲۰}
\resumeItemScore{مدارهای منطقی}{۲۰}
\resumeItemScore{الکترومغناطیس}{۲۰}
\resumeItemScore{برنامه‌نویسی پیشرفته (\lr{AP})}{۱۹.۱۹}
\resumeItemScore{سیگنال‌ها و سیستم‌ها}{۱۸.۴۲}
\resumeItemScore{مبانی ماتریس و جبر خطی}{۱۸.۸}
\resumeItemScore{معماری کامپیوتر و ریزپردازنده‌ها}{۲۰}
\resumeItemScore{شبکه‌های کامپیوتری}{۱۷.۲}
\resumeItemScore{سیستم‌های مخابراتی}{۱۷.۵}
\resumeItemListEnd{}
\end{multicols}

\resumeHeadingListEnd{}
%--------------------------------------------

% The "Technical Skills" section and subsequent sections are intentionally omitted as per your request.
% \section{مهارت‌های فنی} ...

\end{document}