%-------------------------
% Resume LaTeX Template for Mohammad Raziei
% Author : Mohammad Raziei
% Based on various LaTeX resume templates
% License : MIT

% This resume template is designed to showcase professional experience and skills
% in a clean, organized format. Feel free to modify it to suit your needs.
%-------------------------

%------------PACKAGES----------------
\documentclass[a4paper,11pt]{article}

\usepackage{verbatim} % reimplements the "verbatim" and "verbatim*" environments

\usepackage{titlesec} % provides an interface to sectioning commands i.e. custom elements

\usepackage{color} % provides both foreground and background color management

\usepackage{enumitem} % provides control over enumerate, itemize and description

\usepackage{fancyhdr} % provides extensive facilities for constructing headers, footers and also controlling their use

\usepackage{multicol,array}
\usepackage{tabularx} % defines an environment tabularx, extension of "tabular" with an extra designator x, paragraph like column whose width automatically expands to fill the width of the environment

%\usepackage{latexsym} % provides mathematical symbols

%\usepackage{marvosym} % provides martin vogel's symbol font which contains various symbols

\usepackage[empty]{fullpage} % sets margins to one inch and removes headers, footers etc..

\usepackage[hidelinks]{hyperref} % removes color and shadow of hyperlinks

\usepackage[normalem]{ulem} % provides "\ul" (uline) command which will break at line breaks

\usepackage[english]{babel} % provides culturally determined typographical rules for wide range of languages

\usepackage{ragged2e}

%\usepackage{kpfonts}
\usepackage{amsmath,amsthm,mathtools}
\usepackage{fontawesome5}
\usepackage{enumitem}

%-----------------------------------------

%\input{glyphtounicode} % converts glyph names to unicode
%\pdfgentounicode=1 % ensures pdfs generated are ats readable

%-----------------------------------------
\newlist{subitemize}{itemize}{3}
\setlist[subitemize]{label=$\hookleftarrow$,before={\vspace{-4pt}\raggedright\justifying}, resume,leftmargin=*}

%----------FONT OPTIONS-------------------
%\usepackage[default]{sourcesanspro}
\usepackage{graphicx}
\usepackage{wasysym}
\usepackage{textcomp} % uses the font source sans pro
\urlstyle{same} % changes url font from default urlfont to font being used by the document

\usepackage{setspace}


%-----------------------------------------
\usepackage{xepersian}
\settextfont[Scale=1.05]{XB Niloofar} % یا هر فونت فارسی دیگری که نصب کرده‌اید
\setdigitfont{Yas}

\renewcommand{\baselinestretch}{1.2} 



%----------MARGIN OPTIONS-----------------
\pagestyle{fancy} % set page style to one configured by fancyhdr
\fancyhf{} % clear all header and footer fields

\renewcommand{\headrulewidth}{0in} % sets thickness of linerule under header to zero
\renewcommand{\footrulewidth}{0in} % sets thickness of linerule over footer to zero

\setlength{\tabcolsep}{0in} % sets thickness of column separator in tables to zero

% origin of the document is one inch from the top and from and the left
% oddsidemargin and evensidemargin both refer to the left margin
% right margin is indirectly set using oddsidemargin
\addtolength{\oddsidemargin}{-0.5in}
\addtolength{\topmargin}{-0.5in}

\addtolength{\textwidth}{1.0in} % sets width of text area in the page to one inch
\addtolength{\textheight}{1.0in} % sets height of text area in the page to one inch

\raggedbottom{} % makes all pages the height of current page, no extra vertical space added
\raggedright{} % makes all pages the width of current page, no extra horizontal space added
%------------------------------------------


%--------SECTIONING COMMANDS---------
% \titleformat{<command>}
%   [<shape>]{<format>}{<label>}{<sep>}
%     {<before-code>}[<after-code>]

% command is the sectioning command to be redefined
% shape is the style of the font; scshape stands for small caps style
% format is the format to be applied to whole title- label and text; absent here
% label defines the label
% sep is the horizontal separation between label and title body
% before-code is the code to be executed before
% after-code is the code to be executed after

\titleformat{\section}
{\vspace{-5pt}\large\bf}{}
{0em}{\vspace{-6pt}\color{blue}}[\color{black}\titlerule\vspace{-23pt}]
%-------------------------------------


%--------REDEFINITIONS----------------
% redefines the style of the bullet point
\renewcommand\labelitemii{$\vcenter{\hbox{\tiny$\bullet$}}$}

% redefines the underline depth to 2pt
\renewcommand{\ULdepth}{2pt}
%-------------------------------------


%--------CUSTOM COMMANDS--------------
%\vspace{} defines a vertical space of given size, modifying this in custom commands can help stretch or shrink resume to remove or add content

% resumeItem renders a bullet point
\newcommand{\resumeItem}[1]{
\item{\small{#1}}
}
\newcommand{\resumeItemScore}[2]{\item[]\small#1\hfill#2}


% commands to start and end itemization of resumeItem, rightmargin set to 0.11in to avoid the overflow of resumetItem beyond whatever resumeItemHeading is being used
\newcommand{\resumeItemListStart}{\begin{itemize}[rightmargin=0.11in]}
\newcommand{\resumeItemListEnd}{\end{itemize}}

% resumeSectionType renders a bolded type to be used under a section, used as skill type here, middle element is used to keep ":"s in the same vertical line
\newcommand{\resumeSectionType}[3]{
\item\begin{tabular*}{0.96\textwidth}[t]{
p{0.15\linewidth}p{0.02\linewidth}p{0.81\linewidth}
}
\textbf{#1} & #2 & #3
\end{tabular*}\vspace{-2pt}
}

% resumeTrioHeading renders three elements in three columns with second element being italicized and first element bolded, can be used for projects with three elements
\newcommand{\resumeTrioHeading}[3]{
\item\small{
\begin{tabular*}{0.96\textwidth}[t]{
l@{\extracolsep{\fill}}c@{\extracolsep{\fill}}r
}
\textbf{#1} & \textit{#2} & #3
\end{tabular*}
}
}

\newcommand{\resumePantaHeading}[5]{
\item\small{
\begin{tabular*}{0.96\textwidth}[t]{
l@{\extracolsep{\fill}}c@{\extracolsep{\fill}}r
}
\textbf{#1} & #2 & #3 & #4 & #5
\end{tabular*}
}
}

% resumeQuadHeading renders four elements in a two columns with the second row being italicized and first element of first row bolded, can be used for experience and projects with four elements
\newcommand{\resumeQuadHeading}[4]{
\item
\begin{tabular*}{0.96\textwidth}[t]{r@{\extracolsep{\fill}}l}
\textbf{#1} & #2 \\
\textit{\small#3} & \textit{\small #4} \\
\end{tabular*}
}

% resumeQuadHeadingChild renders the second row of resumeQuadHeading, can be used for experience if different roles in the same company need to added
\newcommand{\resumeQuadHeadingChild}[2]{
\item
\begin{tabular*}{0.96\textwidth}[t]{l@{\extracolsep{\fill}}r}
\textbf{\small#1} & {\small#2} \\
\end{tabular*}
}


% commands to start and end itemization of resumeQuadHeading, lefmargin for left indent of 0.15in for resumeItems
\newcommand{\resumeHeadingListStart}{
\begin{itemize}[rightmargin=0.15in, label={}]
}
\newcommand{\resumeHeadingListEnd}{\end{itemize}}
%-------------------------------------------
\newcommand{\vcenteredinclude}[2][none]{\begingroup
\setbox0=\hbox{\includegraphics[#1]{#2}}%
\parbox{\wd0}{\box0}\endgroup}

%__________________RESUME____________________
% You can rearrange sections in any order you may prefer
\begin{document}

%-----------CONTACT DETAILS------------------
% Make sure all the details are correct, you can add more links in the first row of second column if needed

\begin{tabular*}{\textwidth}{r@{\extracolsep{\fill}}l}

\textbf{\Huge محمد رضیئی فیجانی \vspace{2pt}} % Translated Name

&
\vcenteredinclude[width=2.5cm]{images/profile}
\\
\multicolumn{2}{r}{
\begin{tabular}{l@{\quad|\quad}l@{\quad|\quad}l}
\faLocationArrow\hspace{3pt} مکان: تهران، ایران  % row = 2, col = 1
&
\faMobile\hspace{3pt} موبایل: \lr{+989105093143}   % row = 2, col = 1
&
\faTelegram\hspace{3pt} تلگرام: \href{https://t.me/mohammadraziei}{\uline{\lr{@mohammadraziei}}}
\end{tabular}
}
\\
\faLinkedin\hspace{3pt} \uline{\lr{\url{https://www.linkedin.com/in/mohammadraziei/}}} &
\faEnvelope\hspace{3pt} ایمیل: \hspace{18pt} \href{mailto:mohammadraziei1375@gmail.com}{\uline{\lr{mohammadraziei1375@gmail.com}}} % row = 2, col = 2
\\
\faGithub\hspace{3pt} \uline{\lr{\url{https://www.github.com/mohammadraziei/}}} &
\faGlobe\hspace{3pt} وب‌سایت: \uline{\lr{\url{https://mohammadraziei.github.io}}}
\\
\end{tabular*}
%--------------------------------------------




%-----------SUMMARY--------------------------
% Keep this short, simple and straigth to point
\justifying
\section{درباره‌ من} % Translated Section Title
\small{
من یک پژوهشگر داده با انگیزه بالا هستم که در پردازش سیگنال (شامل تصویر، پردازش زبان طبیعی (\lr{NLP}) و صوت)، موازی‌سازی و محاسبات با عملکرد بالا (\lr{HPC}) تخصص دارم. مدارک کارشناسی خود را در رشته‌های سیستم‌های مخابراتی و مهندسی کامپیوتر از دانشگاه صنعتی امیرکبیر و مدرک کارشناسی ارشد را در مهندسی پزشکی از دانشگاه صنعتی شریف دریافت کرده‌ام و تجربه عملی در بازسازی تصاویر \lr{MRI} دارم. در حال حاضر، به عنوان دانشجوی دکترا در رشته مخابرات در دانشگاه صنعتی شریف مشغول به تحصیل هستم و از طریق برنامه نخبگان پذیرفته شده‌ام. تحقیقات دکترای من بر روی کاربرد سطوح هوشمند قابل پیکربندی مجدد (\lr{Reconfigurable Intelligent Surfaces - RIS}) برای بهینه‌سازی محیط‌های بی‌سیم و توسعه الگوریتم‌های پیشرفته پردازش سیگنال با هدف مکان‌یابی دقیق در سیستم‌های ارتباطی نسل آینده متمرکز است. در توسعه بک‌اند (\lr{backend development}) و پیاده‌سازی الگوریتم‌های پیچیده مهارت دارم، در محیط‌های تیمی به خوبی کار می‌کنم، از مهارت‌های ارتباطی قوی برخوردارم و به یادگیری مستمر برای پیشبرد نوآوری و حل چالش‌های پیچیده متعهد هستم.
}
%--------------------------------------------
%-----------EDUCATION-------------------------
% Mention your CGPA, if its good, in the first row of second column
\section{تحصیلات}
\resumeHeadingListStart{}

\resumeQuadHeading{\faGraduationCap\hspace{3pt} دانشگاه صنعتی شریف}{تهران، ایران}
{\textbf{دکتری} مهندسی سیستم‌های مخابراتی - دانشکده مهندسی برق}{$2024$ -- اکنون}
\begin{subitemize}
    \item\textbf{موضوع رساله:}
        \textit{پردازش سیگنال سطوح هوشمند بازپیکربارپذیر (\lr{RIS})}.
    \item[]\textbf{توضیحات:}
        پس از اتمام خدمت سربازی، در سال $2024$ بدون شرکت در آزمون ورودی و از طریق بنیاد ملی نخبگان وارد دوره دکتری شدم. پژوهش من بر فناوری سطوح هوشمند بازپیکربارپذیر (\lr{RIS}) متمرکز است.
    \item[]\textbf{استاد راهنما:} دکتر م. فخارزاده (\href{https://sharif.edu/~fakharzadeh/}{\uline{وب‌سایت}})
\end{subitemize}

% EE SHARIF
\resumeQuadHeading{\faGraduationCap\hspace{3pt} دانشگاه صنعتی شریف}{تهران، ایران}
{\textbf{کارشناسی ارشد} مهندسی پزشکی - دانشکده مهندسی برق}{$2019$ -- $2022$}
\begin{subitemize}
    \item\textbf{عنوان پایان‌نامه:}
        \textit{بازسازی تصاویر \lr{MRI} زیرنمونه‌برداری شده}.
    \item[]\textbf{توضیحات:}
        تسریع تصویربرداری \lr{MRI} با استفاده از \lr{compressed sensing} و تصویربرداری موازی برای بازسازی داده‌های زیرنمونه‌برداری شده. موفق به دستیابی به افزایش سرعت $10$ برابری با حفظ کیفیت تصویر بالا. دفاع موفق (اوایل $2022$)؛ پایان‌نامه تا $5$ سال محرمانه برای انتشار و ثبت اختراع.
    \item[]\textbf{استاد راهنما:} دکتر ب. وثوقی وحدت (\href{https://ee.sharif.edu/~vahdat/}{\uline{وب‌سایت}})
\end{subitemize}
\vspace{.5em}

% CE AUT
\resumeQuadHeading{\faGraduationCap\hspace{3pt} دانشگاه صنعتی امیرکبیر}{تهران، ایران}
{\textbf{کارشناسی} مهندسی نرم‌افزار - دانشکده مهندسی کامپیوتر}{$2017$ -- $2020$}
\begin{subitemize}
    \item\textbf{توضیحات:}
        به دلیل معدل بالا و رتبه $461$ در کنکور سراسری، علاوه بر مهندسی برق، از طریق بنیاد نخبگان دانشگاه در رشته مهندسی کامپیوتر نیز ادامه تحصیل دادم.
\end{subitemize}
\vspace{.5em}

% EE AUT
\resumeQuadHeading{\faGraduationCap\hspace{3pt} دانشگاه صنعتی امیرکبیر}{تهران، ایران}
{\textbf{کارشناسی} مهندسی سیستم‌های مخابراتی - دانشکده مهندسی برق}{$2015$ -- $2019$}
\begin{subitemize}
    \item\textbf{عنوان پایان‌نامه:} \textit{الگوریتم مبتنی بر یادگیری تقویتی برای کنترل وسایل نقلیه خودران}.
    \item[]\textbf{جزئیات پروژه:} توسعه بسته پایتون \lr{Gym-Prescan} برای اتصال \lr{OpenAI Gym} به پلتفرم شبیه‌سازی \lr{PreScan} و انجام آزمایش‌های یادگیری تقویتی برای کنترل خودروهای خودران؛ این ابزار هسته اصلی پایان‌نامه بود.
    \item[]\textbf{افتخارات:}
        این پایان‌نامه در پایان سال $2019$ موفق به کسب رتبه برتر صنعت شد. این رویداد با ارزیابی هیئت داوران متشکل از اساتید برق و داوران صنعتی برگزار شد.
    \item[]\textbf{اساتید راهنما:} دکتر و. پوراحمدی (\href{https://aut.ac.ir/cv/2519/VAHID%20POURAHMADI}{\uline{وب‌سایت}})، دکتر ح. امین‌داور (\href{https://aut.ac.ir/cv/2200/Hamidreza-Amindavar}{\uline{وب‌سایت}})
\end{subitemize}
\resumeHeadingListEnd{}
%--------------------------------------------

\section{برخی نمرات}
\resumeHeadingListStart{}
\resumeQuadHeading{کارشناسی ارشد}{}{}{}\\\vspace{-1.5em}

\begin{multicols}{2}
\resumeItemListStart{}
\resumeItemScore{پردازش سیگنال روی گراف (\lr{GSP})}{$20$}
\resumeItemScore{برنامه‌نویسی موازی}{$20$}
\resumeItemScore{یادگیری عمیق}{$20$}
\resumeItemScore{تصویربرداری پزشکی}{$18.5$}
\resumeItemScore{فشرده‌سازی سنجشی (\lr{CS})}{$18.3$}
\resumeItemScore{تحلیل و پردازش تصاویر پزشکی (\lr{MIAP})}{$18$}
\resumeItemScore{جداسازی منابع کور (\lr{BSS})}{$17.6$}
\resumeItemScore{فیلتر تطبیقی}{$17$}
\resumeItemListEnd{}
\end{multicols}

\resumeQuadHeading{کارشناسی}{}{}{}\\\vspace{-1.5em}
\begin{multicols}{2}
\resumeItemListStart{}
\resumeItemScore{پردازش سیگنال دیجیتال (\lr{DSP})}{$20$}
\resumeItemScore{برنامه‌نویسی کامپیوتر}{$20$}
\resumeItemScore{مدارهای منطقی}{$20$}
\resumeItemScore{الکترومغناطیس}{$20$}
\resumeItemScore{برنامه‌نویسی پیشرفته (\lr{AP})}{$19.19$}
\resumeItemScore{سیگنال‌ها و سیستم‌ها}{$18.42$}
\resumeItemScore{مبانی ماتریس و جبر خطی}{$18.8$}
\resumeItemScore{معماری کامپیوتر و ریزپردازنده‌ها}{$20$}
\resumeItemScore{شبکه‌های کامپیوتری}{$17.2$}
\resumeItemScore{سیستم‌های مخابراتی}{$17.25$}
\resumeItemListEnd{}
\end{multicols}

\resumeHeadingListEnd{}


\section{مهارت‌های فنی}
\resumeHeadingListStart{}
\resumeSectionType{علم داده}{:}{یادگیری ماشین، یادگیری عمیق، یادگیری تقویتی، پردازش زبان طبیعی، شبکه‌های عصبی گراف، پردازش تصویر، پردازش سیگنال}
\resumeSectionType{پردازش سیگنال}{:}{پردازش سیگنال‌های مخابراتی کلاسیک، پردازش سیگنال مبتنی بر هوش مصنوعی، شناسایی خودکار مدولاسیون (\lr{AMR})، پردازش \lr{EEG} و \lr{stereo-EEG} برای بازشناسی گفتار ذهنی}
\resumeSectionType{سیستم‌های مخابراتی}{:}{\lr{5G}، \lr{6G}، سطح هوشمند بازپیکربارپذیر (\lr{RIS})، مکان‌یابی، ارتباطات بی‌سیم}
\resumeSectionType{زبان‌ها}{:}{پایتون، \lr{C++}، \lr{Cuda}، \lr{Matlab}، \lr{JavaScript}}
\resumeSectionType{برنامه‌نویسی}{:}{الگوهای طراحی، توسعه مبتنی بر تست، معماری مایکروسرویس، توسعه بسته، اتصال \lr{CPython}، معماری تمیز}
\resumeSectionType{\lr{devops}}{:}{\lr{git}، \lr{CI/CD}، \lr{gitlab-ci}، \lr{github actions}، \lr{template repo}}
\resumeHeadingListEnd{}


\section{تجربیات منتخب}
\resumeHeadingListStart{}

\resumeQuadHeading{\faBuilding\hspace{3pt} شرکت آرمان}{بهمن ۱۴۰۲ -- اکنون}
{مهندس ارشد یادگیری ماشین}{تهران، ایران}\\
\justifying{
    در نقش مهندس ارشد یادگیری ماشین، تمرکز من بر توسعه و پیاده‌سازی الگوریتم‌های پیشرفته برای پردازش صوت و گفتار است. مسئولیت‌هایم شامل پردازش سیگنال، طراحی مدل‌ها برای وظایفی مانند شناسایی فعالیت صوتی (\lr{VAD})، حذف نویز گفتار، شناسایی زبان و تشخیص جنسیت می‌باشد. همچنین با آموزش و راهنمایی کارآموزان در مفاهیم یادگیری ماشین و کاربردهای عملی، به رشد تیم کمک می‌کنم.
}

\justifying{\textbf{مهارت‌های کلیدی:} یادگیری ماشین، پردازش سیگنال، پردازش صوت، پردازش گفتار، \lr{VAD}، حذف نویز، شناسایی زبان، تشخیص جنسیت، \lr{Python}، منتورینگ}

\resumeQuadHeading{\faBuilding\hspace{3pt} همراه اول - \lr{MCINEXT}}{بهمن ۱۴۰۰ -- بهمن ۱۴۰۲}
{دانشمند داده - توسعه‌دهنده ارشد \lr{Python} و \lr{C++} (خدمت سربازی)}{تهران، ایران}\\
{
فعالیت خود را در تیم موتور جستجوی همراه اول به عنوان بخشی از خدمت سربازی آغاز کردم. حوزه‌های اصلی تمرکزم شامل تحلیل صفحات وب، هوش مصنوعی، تحلیل گراف وب، تبدیل کد \lr{Python} به \lr{C++} و اتصال آن به \lr{Python} بود. در این فرآیند با مهارت‌های \lr{DevOps} آشنا شدم و اصول زیادی از مهندسی نرم‌افزار را فرا گرفتم.
}

{\textbf{مهارت‌های کلیدی:} \lr{Qt}، \lr{C++}، \lr{Boost}، \lr{Binding}، \lr{Python}، تحلیل وب، هوش مصنوعی، تحلیل گراف، \lr{DevOps}، مهندسی نرم‌افزار، بهینه‌سازی کد}

\resumeQuadHeading{\faBuilding\hspace{3pt} شرکت سونیار}{اردیبهشت ۱۳۹۷ -- دی ۱۳۹۹}
{عضو تا سرپرست تیم پردازش سیگنال و مهندسی نرم‌افزار}{تهران، ایران}\\
فعالیت خود را در این شرکت واقع در پارک فناوری پردیس به عنوان عضو تیم‌های پردازش سیگنال و نرم‌افزار آغاز کردم. ابتدا برای تیم نرم‌افزار مصاحبه دادم اما در هر دو بخش مشغول به کار شدم. در تیم پردازش سیگنال، سیگنال‌های مخابراتی را تحلیل و الگوریتم‌های مختلفی را در \lr{MATLAB} پیاده‌سازی کردیم. به عنوان مترجم کدهای \lr{MATLAB} به \lr{C++}، این الگوریتم‌ها را در نرم‌افزار پردازش سیگنال مبتنی بر \lr{Qt} پیاده‌سازی نمودم.

سپس به سرپرست تیم برنامه‌نویسی ارتقا یافتم. پروژه‌های شرکت در حوزه‌های مخابرات و نرم‌افزار، از جمله پروژه‌های هوش مصنوعی بود. این پروژه‌ها شامل وظایفی مانند شناسایی خودکار مدولاسیون (\lr{AMR}) و مانیتورینگ بودند. همچنین پروژه‌هایی در حوزه محاسبات با کارایی بالا با استفاده از \lr{CUDA} برای موازی‌سازی انجام شد.

\textbf{مهارت‌های کلیدی:} \lr{Qt}، \lr{C++}، \lr{Matlab}، \lr{Python}، پردازش سیگنال، مخابرات، شناسایی مدولاسیون، مانیتورینگ، \lr{CUDA}، رهبری تیم، پیاده‌سازی الگوریتم

\resumeQuadHeading{\faBuilding\hspace{3pt} رباتیک پارسیان امیرکبیر}{اسفند ۱۳۹۴ -- آذر ۱۳۹۵}
{عضو تیم هوش مصنوعی}{تهران، ایران}\\
در ترم دوم کارشناسی به دلیل معدل بالا به تیم رباتیک \href{http://www.parsianrobotics.aut.ac.ir/}{\textit{رباتیک پارسیان امیرکبیر}} دعوت شدم. فعالیت خود را در تیم هوش مصنوعی آغاز کردم. تیم رباتیک پارسیان روی ربات‌های فوتبالیست کار می‌کرد و تیم‌های مختلفی داشت. حوزه فعالیت تیم، پیاده‌سازی الگوریتم‌های کنترل تیمی با استفاده از زبان \lr{C++} و فریم‌ورک \lr{Qt 4.8} بود.

\textbf{مهارت‌های کلیدی:} \lr{C++}، هوش مصنوعی، الگوریتم‌ها، \lr{Qt}، رباتیک

\resumeHeadingListEnd{}


\section{تجربه تدریس}
\resumeHeadingListStart{}
\resumeQuadHeading{\faChalkboardTeacher\hspace{3pt} دوره پیشرفته پایتون}{اردیبهشت ۱۴۰۴}
{مدرس، آکادمی همراه}{۱۰ ساعت}
\begin{subitemize}
    \item\textbf{توضیحات:} این دوره جامع به مفاهیم پیشرفته پایتون می‌پردازد؛ از کتابخانه‌های اصلی و برنامه‌نویسی شی‌گرا تا برنامه‌نویسی تابعی، الگوهای طراحی و سرویس‌های وب. شرکت‌کنندگان مهارت توسعه برنامه‌های قدرتمند و انتشار بسته‌های پایتون خود را کسب می‌کنند.
    \item[]\textbf{وب‌سایت:} \href{https://mohammadraziei.github.io/advanced-python-course/}{\uline{مواد دوره}}
\end{subitemize}
\resumeQuadHeading{\faChalkboardTeacher\hspace{3pt} کارگاه \lr{Advanced Python in Data Science}}{بهمن ۱۴۰۳}
{مدرس، \lr{Hamrah Academy}}{۳ ساعت}
\begin{subitemize}
    \item\textbf{توضیحات:} این کارگاه بر تکنیک‌های پیشرفته پایتون ویژه کاربردهای علم داده تمرکز دارد؛ شامل برنامه‌نویسی شی‌گرا پیشرفته، دکوراتورها، تایپ‌گذاری و ساخت رابط‌های تعاملی با \lr{Gradio}.
\end{subitemize}

\resumeQuadHeading{\faChalkboardTeacher\hspace{3pt} کارگاه برنامه‌نویسی پیشرفته \lr{C++} و \lr{Qt}}{تیر ۱۴۰۱}
{مدرس، دانشگاه امیرکبیر}{۶ ساعت}
\begin{subitemize}
    \item\textbf{توضیحات:} این کارگاه به مفاهیم پیشرفته برنامه‌نویسی \lr{C++}، پوشش عمیق \lr{STL}، مبانی فریم‌ورک \lr{Qt} و سایر ویژگی‌های پیشرفته \lr{C++} برای ساخت برنامه‌های قدرتمند می‌پردازد.
\end{subitemize}

\resumeQuadHeading{\faChalkboardTeacher\hspace{3pt} دوره برنامه‌نویسی مقدماتی (زبان \lr{C})}{بهمن ۱۳۹۹}
{دستیار آموزشی، دانشگاه صنعتی شریف}{}
\begin{subitemize}
    \item\textbf{مدرس:} دکتر ب. وثوقی وحدت
    \item[]\textbf{توضیحات:} همکاری در تدریس مفاهیم مقدماتی برنامه‌نویسی با زبان \lr{C}، شامل اصول پایه برنامه‌نویسی و حل مسئله.
\end{subitemize}
\resumeHeadingListEnd{}




\section{انتشارات}
\resumeHeadingListStart{}
% ---- Book ----
\item \faBook\hspace{3pt} \textbf{یکپارچه‌سازی \lr{Simulink}، \lr{MATLAB} و \lr{Python} با استفاده از \lr{MATLAB Engine API}}. (تکمیل شده، در جستجوی ناشر)\\
\textit{محمد رضیئی فیجانی} (\href{https://mohammadraziei.github.io}{homepage})\\
راهنمای عملی اتصال \lr{Simulink}، \lr{MATLAB} و \lr{Python} با استفاده از \lr{MATLAB Engine API}، شامل تکنیک‌های یکپارچه‌سازی و مثال‌های واقعی برای مهندسان و پژوهشگران.

\item \faBook\hspace{3pt} \textbf{مبانی پردازش سیگنال روی گراف}. (در حال نگارش)\\
\textit{آرش امینی} (\href{https://ee.sharif.edu/~amini/}{homepage})، \textit{محمد رضیئی فیجانی} (\href{https://mohammadraziei.github.io}{homepage})\\
\textbf{وب‌سایت:} \href{https://mohammadraziei.github.io/GSPbook}{mohammadraziei.github.io/GSPbook}\\
این کتاب به مبانی پردازش سیگنال روی گراف می‌پردازد و مرجع اصلی درس مرتبط در دانشگاه صنعتی شریف خواهد بود.

\item \faBook\hspace{3pt} \textbf{کاربرد یادگیری تقویتی در کنترل خودروهای خودران}.\\
\textit{محمد رضیئی فیجانی} (\href{https://mohammadraziei.github.io}{homepage})\\
انتشارات دانشگاه فرهمند، ۲۰۲۱.\\
این کتاب به کاربرد الگوریتم‌های یادگیری تقویتی در کنترل خودروهای خودران می‌پردازد و مفاهیم پایه، چالش‌های عملی و مطالعات موردی را پوشش می‌دهد.\\
\textbf{لینک:} \href{https://farbook.ir/BookView/61151/%d9%83%d8%a7%d8%b1%d8%a8%d8%b1%d8%af-%d9%8a%d8%a7%d8%af%da%af%d9%8a%d8%b1%d9%8a-%d8%aa%d9%82%d9%88%d9%8a%d8%aa%d9%8a-%d8%af%d8%b1-%d9%83%d9%86%d8%aa%d8%b1%d9%84-%d8%a7%d8%aa%d9%88%d9%85%d8%a8%d9%8a%d9%84-%d9%87%d8%a7%d9%8a-%d8%ae%d9%88%d8%af%d8%b1%d8%a7%d9%86}{\uline{لینک ناشر}}

% ---- Journal Paper ----
\item \faFile\hspace{3pt} \textbf{\lr{ESACT}: رویکرد بازآرایی اندیس برای حذف تداخل بانک در الگوریتم‌های جمع پیشوندی موازی بدون پدینگ حافظه اشتراکی}. ارسال شده به \textit{Journal of Parallel and Distributed Computing}، ۲۰۲۴.\\
\textit{محمد رضیئی فیجانی} (\href{https://mohammadraziei.github.io}{homepage})، 
\textit{رضا کاظمی} (\href{https://sina.sharif.edu/~reza.kazemi/}{homepage})، 
\textit{متین هاشمی} (\href{https://sina.sharif.edu/~matin/}{homepage})\\
\textbf{چکیده:} تداخل بانک به طور قابل توجهی عملکرد الگوریتم‌های موازی روی معماری \lr{GPU} را با سریالی‌سازی دسترسی به حافظه کاهش می‌دهد. این مشکل به ویژه در محاسبات جمع پیشوندی موازی (اسکن) تأثیرگذار است... ما \lr{ESACT} را معرفی می‌کنیم؛ یک الگوریتم جدید جمع پیشوندی موازی که بدون نیاز به پدینگ حافظه، تداخل بانک را حذف می‌کند. \lr{ESACT} الگوریتم \lr{Blelloch} را به یک طرح آدرس‌دهی ترتیبی با اندیس‌های بازآرایی‌شده بهینه تبدیل می‌کند...
\resumeHeadingListEnd{}

\section{سرگرمی‌ها}
\resumeItemListStart{}
\resumeItem{مطالعه، حل روبیک، گوش دادن به موسیقی، تماشای فیلم، سفر، بازی چندنفره، گاهی کدنویسی فرانت‌اند برای سرگرمی}
\resumeItemListEnd{}

\end{document}